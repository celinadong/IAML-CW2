%%%%%%%%%%%%%%%%%%%%%%%%%%%%%%%%%%%%%%%%%%%%%%%%%%%%%%%%
%                       Assignment 2                   %
%                                                      %
% Author: Michael P. J. Camilleri					   %
%                                                      %
% Based on the Cleese Assignment Template for Students %
% from http://www.LaTeXTemplates.com.				   %
%                                                      %
% Original Author: Vel (vel@LaTeXTemplates.com)		   %
%													   %
% License:											   %
% CC BY-NC-SA 3.0 									   %
% (http://creativecommons.org/licenses/by-nc-sa/3.0/)  %
% 													   %
%%%%%%%%%%%%%%%%%%%%%%%%%%%%%%%%%%%%%%%%%%%%%%%%%%%%%%%%

%--------------------------------------------------------
%   IMPORTANT: Do not touch anything in this part
\documentclass[12pt]{article}
\input{style.tex}



% Options for Formatting Output

\global\setbool{clearon}{true} %
\global\setbool{authoron}{true} %



\newcommand{\assignmentQuestionName}{Question}
\newcommand{\assignmentTitle}{Assignment\ \#2}

\newcommand{\assignmentClass}{IAML -- INFR10069 (LEVEL 10)}

\newcommand{\assignmentWarning}{NO LATE SUBMISSIONS} % 
\newcommand{\assignmentDueDate}{Friday,\ November\ 15,\ 2019 @ 16:00}
%--------------------------------------------------------

%--------------------------------------------------------
%   IMPORTANT: Specify your Student ID below [You will need to uncomment the line, else compilation will fail]. Make sure to specify your student ID correctly, otherwise we may not be able to identify your work and you will be marked as missing.
\newcommand{\assignmentAuthorName}{s1701688}
%--------------------------------------------------------

\begin{document}
\maketitle
\thispagestyle{empty}



%%%%%%%%%%%%%%%%%%%%%%%%%%%%%%%%%%%%%%%%%%%%%%%%%%%%%%%%%%%%%%%%%%%%%%%%%%%%%%
%============================================================================%
%%%%%%%%%%%%%%%%%%%%%%%%%%%%%%%%%%%%%%%%%%%%%%%%%%%%%%%%%%%%%%%%%%%%%%%%%%%%%%


\assignmentSection{Part A: 20-NewsGroups [60 Points]}




\begin{question}{(10 points) Exploratory Analysis}

\questiontext{We will begin by exploring the Dataset to get some insight about it.}



\begin{subquestion}{(5 points) Focusing first on the training set, summarise the key features/observations in the data: focus on the dimensionality, data ranges, feature and class distribution and report anything out of the ordinary. What are the typical values of the features like?}


\answerbox{12em}{
The training set comprises of 5648 instances and 1000 attributes (features). We observe that the data ranges from 0 to 1 given that the minimum value for all the attributes is 0 and the maximum never surpasses 1. All the attributes are non-null, and all of them are of type float64. From the data obtained by the describe() function, we see that the mean value for each attribute is quite low, although they seem like they're greater than the median. Also, it seems that the values for all the percentiles are 0. By comparing the 75th percentile against the max value for each of the attributes, we could deduce that there exists outliers in the data set. The distribution of the classes is overall quite balanced, except for the class 7, talk.religion.misc, which has around 1/3 less instances than the other classes.
We observe that the classes 0 and 1 are closely related (comp), as well as 2 and 3 (rec), 4 and 5 (sci), 6 and 7 (religion). By applying the .unique() function to the dataset, we observe that there are only 110 different unique values in our dataset, and using the .value\_counts() functions it seems that all of those values only have 1 count except for 0. In fact, the typical values of the features is 0; 5539 of our instances take value 0 in our dataset.
}



\end{subquestion}


\begin{subquestion}{(3 points) Looking now at the Testing set, how does it compare with the Training Set (in terms of sizes and feature-distributions) and what could be the repurcussions of this?}


\answerbox{10em}{
In the testing set, we have less instances than in the training set, 1883 instances. The number of attributes remains the same. All the values are non-null of type float64. We have 32 unique values, 32/1883=0.017, the proportion of unique values is slightly lower than the one we have for our training dataset 110/5648=0.019. In general, discarding the value 0, we observe that the values in training dataset are different to those in testing dataset. When splitting the training and testing datasets, it is preferred that the testing dataset contains instances that our model has never seen before. This way, we will know if our model has overfit to our training set, and the comparison of its performance on the testing set. If the performance on the testing set is relatively good and does not differ by much from that of the training set, then we can say our model generalises data quite well. Like in the training dataset, there are less instances labeled 7.
}



\end{subquestion}

\begin{subquestion}{(2 points) Why do you think it is useful to consider TF-IDF weights as opposed to just the frequency of times a word appears in a document as a feature?}



\answerbox{10em}{
TF-IDF balances out the frequency of words by how often they appear across all documents, it reflects how relevant a word in a document is. It might be the case that the word is really frequent in general, such as 'the' or other stopwords, but are not really relevant or meaningful. TF-IDF reduces the weight of these common meaningless words and gives higher weight to uncommon words in a document. For instance, a document about Business would highlight words like 'enterprise' and 'stakeholder'. If we used the number of times a word appears in a document, then we would get that extremely frequent words dominate in some documents, even though they do not provide any useful information to our model.
}



\end{subquestion}



\end{question}


%============================================================================%

\begin{question}{\label{Q_UNSUP_LEARN}(24 points) Unsupervised Learning}

\questiontext{We will now explore the documents in some detail by way of clustering.}



\begin{subquestion}{(2 points) The K-Means algorithm is non-deterministic. Explain why this is, and how the final model is selected in the SKLearn implementation of \href{https://scikit-learn.org/stable/modules/clustering.html}{KMeans}.}



\answerbox{8em}{
K-Means algorithm is non-deterministic because if we were to run the algorithm on the same dataset various times, we could obtain different results and classifications. That is, the algorithm starts with an initial set of k cluster centres randomly selected, classifies the data points according to these centroids and then recomputes the centroids. Given that the initial centroids for the algorithm are randomly selected, on different executions of the algorithm, we get different outputs for the same dataset inputted. This makes the results obtained hard to compare.

The SKLearn K-means algorithm stops when the difference between the previous and new centroids is less than a specific threshold. It uses the threshold as a stopping criterion for the algorithm, and runs until the centroids of the clusters do not change significantly. The final model is the one obtained when we stop after reaching the threshold.
}



\end{subquestion}


\begin{subquestion}{(1 point) One of the parameters we need to specify when using k-means is the number of clusters. What is a reasonable number for this problem and why?}



\answerbox{5em}{
A reasonable number of clusters for our newsgroups problem is the number of distinct classes we have in the dataset, that is, 8. All our instances are classified to one of the classes which we can find in the news\_labels dataframe. By using the number of classes as the number of clusters for k-means, we are clustering the instances in 8 different clusters (each corresponding to one news label). This way, we will obtain the label that corresponds to each data point.
}



\end{subquestion}


\begin{subquestion}{(5 points) We will use the Adjusted Mutual Information (AMI) \ie \href{https://scikit-learn.org/stable/modules/clustering.html\#mutual-info-score}{\texttt{adjusted\_mutual\\\_info\_score}} between the clusters and the true (known) labels to quantify the performance of the clustering. Give an expression for the MI in terms of entropy. In short, describe what the MI measures about two variables, why this is applicable here and why it might be difficult to use in practice. \hint{MI is sometimes referred to as Information Gain: note that you are asked only about the standard way we defined MI and not the AMI which is adjusted for the size of the domain and for chance agreement.}}



\answerbox{16em}{
Expression for the MI in terms of entropy is MI(U,V) = H(U) + H(V) - H(U,V)

MI measures how accurate the clustering algorithm's label predictions are compared to the actual true labels, that is, the agreement of the two assignments ignoring permutations (how related to variables are to each other). It provides a general measure based on the joint probabilities of two variables assuming no underlying relationship (like linearity). MI is capable of identifying more information about the relationship of two variables than correlation can.

It is applicable here because it measures the similarity between the clusters and true classes, so we can determine how well the clustering algorithm performed. In practice, this metric might be difficult to use because we might not always have the ground truth labels of the instances, so we cannot compare the predicted labels against these. In general, we use unsupervised learning to determine the clusters and possible classifications of our data when we do not have labels.
}



\end{subquestion}

\begin{subquestion}{(4 points) Fit K-Means objects with \texttt{n\_clusters} ranging from 2 to 12. Set the random seed to 1000 and the number of initialisations to 50, but leave all other values at default. For each fit compute the adjusted mutual information (there is an SKLearn \href{https://scikit-learn.org/stable/modules/generated/sklearn.metrics.adjusted_mutual_info_score.html}{function} for that). Set \texttt{average\_method=`max'}. Plot the AMI scores against the number of clusters (as a line plot).}



\answerbox{40em}{
\begin{center}
 \includegraphics[width=0.9\textwidth]{graphs/24.png}
 \end{center}
}



\end{subquestion}

\begin{subquestion}{\label{Q_CLUSTER_TRENDS}(3 points) Discuss any trends and interesting aspects which emerge from the plot. Does this follow from your expectations?}



\answerbox{10em}{
From the lineplot, we observe that for 2 clusters, the AMI score is quite low (0.0983) and the score increases rapidly as we increase the number of clusters. However, we observe that the AMI for 4 clusters is unusually higher than the score for 5. For 9 clusters, we get the maximum AMI score of 0.3343. According to our plot, 9 clusters would be the ideal number for our clustering because it achieves the highest performance. But as we keep increasing the number of clusters, the AMI decreases slowly because we reach a point where the centroids start to overlap and each datapoint is assigned to one centroid (itself).

I was expecting to have a low AMI with a small number of clusters and a specific number of clusters where the AMI reaches the maximum score and starts decreasing after that peak. But I was not expecting to have a higher score for 4 clusters than for 5.
}



\end{subquestion}

\begin{subquestion}{\label{Q_CLUSTER_FOUR}(6 points) Let us investigate the case with four (4) clusters in some more detail. Using seaborn's \href{https://seaborn.pydata.org/generated/seaborn.countplot.html}{\texttt{countplot}} function, plot a bar-chart of the number of data-points with a particular class (encoded by colour) assigned to each cluster centre (encoded by position on the plot's x-axis). As part of the cluster labels, include the total number of data-points assigned to that cluster.}



\answerbox{40em}{
\begin{center}
 \includegraphics[width=0.9\textwidth]{graphs/26.png}
 \end{center}
}



\end{subquestion}

\begin{subquestion}{(3 points) How does the clustering in Question\ref{Q_UNSUP_LEARN}:\ref{Q_CLUSTER_FOUR} align with the true class labels? Does it conform to your observations in Question\ref{Q_UNSUP_LEARN}:\ref{Q_CLUSTER_TRENDS}?}



\answerbox{14em}{
The clustering and the true labels do not align very well with their true classes. Both the red and blue bars belong to the bigger class 'comp', and we observe that both of these are grouped together in cluster 1. As for cluster 2, we observe that it clearly aligns with 'sci.crypt' class. Regarding cluster 3, most of the datapoints belong to 'soc.religion.christian' class. In cluster 4, although the 'rec' classes dominate the cluster, we see that quite a lot of datapoints of class 'sci.electronics' are assigned to this cluster. So, we observe that most of the instances are assigned to clusters with their corresponding headers (comp, sci, rec...) but are not aligned with their actual correct classes. This is reflected in our 2.5 graph, where the AMI score for 4 clusters is higher than for 5.
}



\end{subquestion}



\end{question}

%============================================================================%

\begin{question}{(26 points) Logistic Regression Classification}
\label{Q_LR_NG}
\questiontext{We will now try out supervised classification on this data. We will focus on Logistic Regression and measure performance in terms of the \href{https://scikit-learn.org/stable/modules/generated/sklearn.metrics.f1_score.html}{F1} score (familiarise yourself with this score which is related to the precision and recall scores that we learnt about in class).}



\begin{subquestion}{(3 points) What is the F1-score, and why is it preferable to accuracy in our problem? How does the macro-average work to extend the score to multi-class classification?}



\answerbox{8em}{
F1-score is a weighted average of the precision and recall (harmonic mean of the two). Accuracy mainly measures the correctly classified instances, and is largely contributed by True Negatives. In our problem, this can be harmful because we might never classify any instances to class 7, but the model should try to minimize these False Negatives. Thus, F1-score is more appropriate because it measures the misclassified instances better than accuracy and penalizes extreme values. In our news classification problem FN and FP are more important than TP and TN, especially because the classes are imbalanced (especially class 7), so F1-score is a more appropriate metric to use.

The macro-average in a multiclass problem setting computes the F1-score independently for each class and then takes the average, treating all the classes equally without taking into account imbalanced classes, which can result in a misleading F1-score result.
}



\end{subquestion}


\begin{subquestion}{(2 points) As always we start with a simple baseline classifier. Define such a classifier (indicating why you chose it) and report its performance on the \textbf{Test} set. Use the `macro' average for the \texttt{f1\_score}.} %\hint{For the baseline, the classifier should use only the target labels.}



\answerbox{8em}{
I chose a classifier that classifies all instances to the most probable class, which is the class most of the instances in the training dataset are classified to, as my baseline. This seems like a reasonable classifier because if we classify all instances to the most common class (class with most observations), we will make less misclassifications. As we can observe from the F1-score obtained (0.0292), the performance of this classifier on the test set is really bad.
}



\end{subquestion}

\begin{subquestion}{(3 points) We will now train a \href{https://scikit-learn.org/stable/modules/generated/sklearn.linear_model.LogisticRegression.html}{LogisticRegression} Classifier from SKLearn. By referring to the documentation, explain how the Logistic Regression model can be applied to classify multi-class labels as in our case. \hint{Limit your explanation to methods we discussed in the lectures.}}



\answerbox{9em}{
Muticlass classification can be achieved with Logistic Regression by using the one-vs-rest classification, which is a better approach than softmax or sigmoid, for multiclass. In OvR, a Logistic Regression classifier is trained for each class c in our dataset to predict the probability that y=c. So it turns out to be a binary classification where we have a separate binary classifier for each class c, and we calculate whether the datapoing belongs to that class or not.
}



\end{subquestion}

\begin{subquestion}{(4 points) Train a Logistic Regressor on the training data. Set \texttt{solver=`lbfgs'}, \texttt{multi\_class=`multinomial'} and \texttt{random\_state=0}. Use the Cross-Validation object you created and report the average validation-set F1-score as well as the standard deviation. Comment on the result.}



\answerbox{9em}{
Your Answer Here
}



\end{subquestion}

\begin{subquestion}{\label{Q_LOG_REG_PLT}(5 points) We will now optimise the Regularisation parameter $C$ using cross-validation. Train a logistic regressor for different values of $C$: in each case, evaluate the F1 score on the training and validation portion of the fold. That is, for each value of $C$ you must provide the training set and validation-set scores per fold and then compute (and store) the average of both over all folds. Finally plot the (average) training and validation-set scores as a function of $C$. \hint{Use a logarithmic scale for $C$, spanning 19 samples between $10^{-4}$ to $10^5$.}}



\answerbox{40em}{
\begin{center}
 \includegraphics[width=0.9\textwidth]{graphs/35.png}
 \end{center}
}



\end{subquestion}

\begin{subquestion}{(7 points) What is the optimal value of $C$ (and the corresponding score)? How did you choose this value? By making reference to the effect of the regularisation parameter $C$ on the optimisation, explain what is happening in your plot from Question \ref{Q_LR_NG}:\ref{Q_LOG_REG_PLT} \hint{Refer to the documentation for $C$ in the \href{https://scikit-learn.org/stable/modules/generated/sklearn.linear_model.LogisticRegression.html}{LogisticRegression} page on SKLearn}.}



\answerbox{11em}{
The optimal value of parameter C is 1.0, since this is the value of C for which we obtain the highest average F1-score (0.6768). This means that our Logistic Regressor generalises the best for C=1.0. The smaller the value of parameter C, the stronger the regularisation and the larger the value, the weaker the regularisation.

The aim with regularisation is to reduce the generalisation error, and not the training error. So, regularisation generalises better on unseen data and reduces overfitting of the model to the training dataset. I chose value C=1.0 because for this value, the model generalises the best on the validation set. So, although we obtain higher F1-scores for higher parameter C values, we see that the model overfits to the training dataset and does not generalise well on unseen data (validation set). In fact, we see that it overfits because the difference on the F1-score between the training and validation sets keep increasing once we reach the peak for validation at C=1.0. Thus, the optimal value is for the C that has the highest score.
}



\end{subquestion}

\begin{subquestion}{(2 points) Finally, report the score of the best model on the test-set, after retraining on the entire training set (that is drop the folds). \hint{You may need to set \texttt{max\_iter = 200}.} Comment briefly on the result.}



\answerbox{7em}{
The F1-score obtained on the actual test set is 0.6748. This is expected given that the value we obtained on the validation set for C=1.0 during our cross validation was approximately 0.6768. So, the value we get when we train our model on the testing set reflects our F1-score obtained during our cross validation on the validation set.
}



\end{subquestion}


\end{question}




%============================================================================%




%%%%%%%%%%%%%%%%%%%%%%%%%%%%%%%%%%%%%%%%%%%%%%%%%%%%%%%%%%%%%%%%%%%%%%%%%%%%%%
%============================================================================%
%%%%%%%%%%%%%%%%%%%%%%%%%%%%%%%%%%%%%%%%%%%%%%%%%%%%%%%%%%%%%%%%%%%%%%%%%%%%%%

\clearpage

\assignmentSection{Part B: Bristol Air-Quality [90 points]}




\begin{question}{\label{Q_EXPLORATORY}(30 Points) Exploratory Analysis}

\questiontext{We will begin by exploring the Dataset to familiarise ourselves with it.}



\begin{subquestion}{(6 points) Summarise the key features/observations in the data: describe the purpose of each column and report (briefly) also on the dimensionality/ranges (ballpark figures only, and how they compare across features) and number of sites, and identify anything out of the ordinary/problematic: \ie look out for missing data and negative values. Why are the latter unreasonable in such a dataset? \hint{Refer to the documentation for how to interpret the pollutant values.}}



\answerbox{13em}{
Your answer here
}



\end{subquestion}

\begin{subquestion}{(6 points) Repeat the same analysis but this time on a per-site basis. Provide a table with the number of samples and percentage of problematic samples (negative and missing) in each site. To report numbers, count a row which has at least one missing entry
as having missing data, and similarly for negative entries. \hint{Pandas has a handy method, \texttt{to\_latex()}, for generating a latex table from a dataframe.}}



\answerbox{17em}{
Your Table Here
}



\end{subquestion}

\begin{subquestion}{(4 points) Briefly summarise how the sites compare in terms of number of samples and amount of problematic samples.}



\answerbox{11em}{
Your Answer Here
}



\end{subquestion}

\begin{subquestion}{(3 points) Given that the columns are all oxides of nitrogen and hence we expect them to be related, we will now look at correlations in our data. This will also be useful in determining how well we can predict any one of the readings from the other two. Remove the data from sites 3 and 15 and compute the \textbf{Pearson} correlation coefficient between each of the three pollutant columns on the remaining data. Visualise the coefficients between each pair of columns in a table.}



\answerbox{10em}{
Your Table Here
}



\end{subquestion}

\begin{subquestion}{(2 points) Comment on the level of correlation between each pair of pollutants.}



\answerbox{7em}{
Your Answer Here
}



\end{subquestion}



\begin{subquestion}{\label{CORRELATIONS}(5 points) For each of the three pollutants, compute the Pearson correlation between sites. \hint{You will need to remove the `Date Time' column and then group by the first level of the columns.} Then plot these as three heatmaps: show the values within the figures. \hint{Use the method \texttt{plot\_matrix()} from \texttt{mpctools.extensions.mplext}.}}



\answerbox{40em}{
Your Image Here
}



\end{subquestion}

\begin{subquestion}{(4 points) Comment briefly on your observations from Question \ref{Q_EXPLORATORY}:\ref{CORRELATIONS}: start by summarising the results from the NO gas and then comment on whether the same is observed in the other gases or if there is something different.}



\answerbox{12em}{
Your Answer Here
}



\end{subquestion}

\end{question}


%============================================================================%

\begin{question}{(19 Points) Principal Component Analysis}

\questiontext{One aspect which we have not yet explored is the temporal nature of the data. That is, we need to keep in mind that the readings have a temporal aspect to them which can provide some interesting insight. We will explore this next.}



\begin{subquestion}{(1 point) Plot the first 5 lines of data (plot each row as a single line-plot).}



\answerbox{40em}{
Your Image Here
}



\end{subquestion}



\begin{subquestion}{(5 points) We will focus first on data solely from Site 1. Extract the data from this site, and run PCA with the number of components set to 72 for now. Set the \texttt{random\_state=0}. On a single graph plot: (i) the percentage of the variance explained by each principal component (as a bar-chart), (ii) the cumulative variance (line-plot) explained by the first $n$ components: (\hint{you should use \href{https://matplotlib.org/3.1.1/api/_as_gen/matplotlib.axes.Axes.twinx.html}{\texttt{twinx()}} to make the plot fit}), \textsl{and}, (iii) mark the point at which the number of components collectively explain at least 95\% of the variance (using a vertical line). \hint{Number components starting from 1.}}



\answerbox{40em}{
Your Image Here
}



\end{subquestion}

\begin{subquestion}{(2 points) Interpret and summarise the above plot.}



\answerbox{9em}{
Your Answer Here
}



\end{subquestion}


\begin{subquestion}{(5 points) Generate three figures, one for the mean and one for each of the first 2 principal components: in each, plot the mean/component as three lines, one for each pollutant throught one day cycle. \hint{You will need to reshape the components appropriately.}}



\answerbox{50em}{
Your Image Here
}



\end{subquestion}

\begin{subquestion}{(6 points) Focusing on the mean and first principal component, are there any significant patterns which emerge throughout the day? \hint{Think about car usage throughout the day.} What is different when interpreting the mean versus the first component? \hint{Do peaks signify the same thing in both cases?} Looking at the principal components only, are there any significant differences between the pollutants? Why could this be happening? \hint{You can refer to one of the limitations of PCA.}}



\answerbox{16em}{
Your Answer Here
}



\end{subquestion}

\end{question}

%============================================================================%


\begin{question}{\label{Q_LR_BA}(41 points) Regression}


\questiontext{Given our understanding of the correlation between signals and sites, we will now attempt to predict the NOx level for Site 17 given the value at the other sites. We will evaluate our models using the Root Mean Squared Error (RMSE) \ie the square root of the \href{https://scikit-learn.org/stable/modules/generated/sklearn.metrics.mean_squared_error.html}{mean\_squared\_error} score by sklearn.}



\begin{subquestion}{(2 points) First things first: since we are dealing with a supervised task, we will need to split our data into a training and testing set. Furthermore, since some of our regressors will involve hyper-parameter tuning, we will also need a validation set. Use the \texttt{multi\_way\_split()} method from \texttt{mpctools.extensions.skext} to split the data into a Training (60\%), Validation (15\%) and Testing (25\%) set: use the \href{https://scikit-learn.org/stable/modules/generated/sklearn.model_selection.ShuffleSplit.html}{ShuffleSplit} object from sklearn for the \texttt{splitter}. Set the random state to 0. \hint{The method gives you the indices of the split for each set, which can then be applied to multiple matrices.} Report the sizes of each dataset.}



\answerbox{4em}{
Your Answer Here
}



\end{subquestion}

\begin{subquestion}{(4 points) Let us start with a baseline. By using only the $y$-values, what baseline regressor can you define (indicate what it does)? Implement it and report the RMSE on the training and validation sets. Interpret this relative to the statistics of the data.}



\answerbox{8em}{
Your Answer Here
}



\end{subquestion}

\begin{subquestion}{(3 points) Let us now try a more interesting algorithm: specifically, we will start with \href{https://scikit-learn.org/stable/modules/generated/sklearn.linear_model.LinearRegression.html}{LinearRegression}. Train the regressor on the training data and report the RMSE on the training and validation set, and comment on the relative performance to the baseline.}



\answerbox{7em}{
Your Answer Here
}



\end{subquestion}



\begin{subquestion}{(5 points) We want to explore further what the model is learning. Explain why in Linear Regression, we cannot just blindly use the weights of the regression coefficients to evaluate the relative importance of each feature, but rather we have to normalise the features. By referring to the documentation for the \href{http://scikit-learn.org/stable/modules/generated/sklearn.linear_model.LinearRegression.html}{LinearRegression} implementation in SKLearn, explain what the normalisation does and how it helps in comparing features. Will this affect the performance of the Linear Regressor?}



\answerbox{10em}{
Your Answer Here
}



\end{subquestion}

\begin{subquestion}{(5 points) Retrain the regressor, setting \texttt{normalize=True} and report (in a table) the ratio of the relative importance of each feature. Which is the most/least important site? How do they compare with the correlation coefficients for Site 17 as computed in Question \ref{Q_EXPLORATORY}:\ref{CORRELATIONS}, and why do you think that is?}



\answerbox{15em}{
Your Answer Here
}



\end{subquestion}

\begin{subquestion}{(5 points) It might be that with non-linear models, we may get better performance. Let us try to use \href{https://scikit-learn.org/stable/modules/generated/sklearn.neighbors.KNeighborsRegressor.html}{K-Nearest-Neighbours}. Train a KNN regressor with default parameters on the training set and report performance on the training and validation set. \hint{it might be beneficial to set \texttt{n\_jobs=-1} to improve performance.} How does it compare with Linear Regression in terms of performance on both sets? What is a limitation of the KNN algorithm for our dataset?}



\answerbox{8em}{
Your Answer Here
}



\end{subquestion}

\begin{subquestion}{(4 points) The KNN regression allows setting a number of hyper-parameters. We will optimise only one: the number of neighbours to use. By using the validation set, find the optimal value for the \texttt{n\_neighbours} parameter out of the values [2, 4, 8, 16, 32]. Plot the training/validation RMSE and indicate (for example with a line) the best value for \texttt{n\_neighbours}.}



\answerbox{40em}{
Your Image Here
}



\end{subquestion}

\begin{subquestion}{(1 points) What is the best-case RMSE performance on the validation set for KNN?}



\answerbox{6em}{
Your Answer Here
}



\end{subquestion}

\begin{subquestion}{(4 points) Let us try one last regression algorithm: we will now use \href{https://scikit-learn.org/stable/modules/generated/sklearn.tree.DecisionTreeRegressor.html}{DecisionTreeRegressor}. Again, the algorithm contains a number of hyper-parameters, and we will optimise the depth of the tree. Train a series of Decision Tree Regressors, optimising (over the validation set) the \texttt{max\_depth} over the values [2, 4, 8, 16, 32, 64]. Set \texttt{random\_state=0}. Plot the training/validation RMSE and indicate (as before) the best value for \texttt{max\_depth}.}



\answerbox{40em}{
Your Image Here
}



\end{subquestion}

\begin{subquestion}{(3 points) What is the best-case RMSE performance on the validation set? What do you notice from the plot about the performance of the Decision Tree Regressor?}



\answerbox{6em}{
Your Answer Here
}



\end{subquestion}

\begin{subquestion}{(5 points) To conclude let us now compare all the models on the testing set. Combine the training and validation sets and retrain the model from each family on it: in cases where we optimised hyper-parameters, set this to the best-case value. Report the testing-set performance of each model in a table \hint{You should have 4 values}.}



\answerbox{6em}{
Your Answer Here
}



\end{subquestion}

\end{question}

%============================================================================


\end{document}
